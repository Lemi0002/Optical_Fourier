\section{Anwendungen}

% \begin{frame}<1-3>[label=opt:patternMatching]{Anwendungen - Pattern matching}
%     \begin{block}<1->{Muster im Spektrum detektieren}
%         \begin{itemize}
%             \item<1-> Jeden Input gibt eine Fouriertransformation
%             \item<1-> Unterschiedliche Muster - unterschiedliches Spektrum
%             \item<2-> Maskieren des gewünschten Spektrums
%             \item<3-> Helligkeitsmessung nach der Maske
%         \end{itemize}
%     \end{block}
%     \begin{block}<4->{Optik statt Elektronik}
%         \begin{itemize}
%             \item<4-> Linsen statt Prozessor
%             \item<5-> $60 cm$ Distanz plus $100 ps$ Anstiegszeit der Photodiode
%             \begin{itemize}
%                 \item $\frac{0.6 m \cdot s^{-1}}{3 \cdot 10^8 m} = 2 ns$
%                 \item $\frac{1}{2.1 ns = 1 ns} \approx 0.5 GHz$
%             \end{itemize}
%         \end{itemize}
%     \end{block}
% \end{frame}

% \begin{frame}{Anwendungen - Pattern matching}
%     \centering
%     \includegraphics[width = 0.8\textwidth]{../../SeminarHarmonischeAnalysis/buch/papers/opt/images/pattern_YT.png}
%     \vfill
%     Youtube: Huygens Optics \cite{opt:YT:PatternRecognition}
% \end{frame}

% \againframe<4->{opt:patternMatching}

\begin{frame}{Anwendungen - Diffractive deep neural network}
    \centering
    \includegraphics[width = 0.8\textwidth]{../../SeminarHarmonischeAnalysis/buch/papers/opt/images/handwriting_xing.png}
    \vfill
    Xing et al. \cite{opt:Lin.2018}
\end{frame}

\begin{frame}{Anwendungen - Diffractive deep neural network}
    \centering
    \begin{columns}
        \column{0.4\textwidth}
        \includegraphics[width = 1\textwidth]{../../SeminarHarmonischeAnalysis/buch/papers/opt/images/handwriting_5_input_xing.png}
        Input
        \column{0.4\textwidth}
        \includegraphics[width = 1\textwidth]{../../SeminarHarmonischeAnalysis/buch/papers/opt/images/handwriting_5_output_xing.png}
        Output
    \end{columns}
\end{frame}

% \begin{frame}<1->[label=opt:deepLearning]{Anwendungen - Diffractive deep neural network}
%     \begin{block}<1->{Aktuelle Forschung}
%         \begin{itemize}
%             \item<1-> Handschrifterkennung mit Beugung
%             \item<1-> Fünf Beugungsebenen
%             \item<1-> Zehn Detektoren
%             \item<1-> 90\% Erfolgsrate
%         \end{itemize}
%     \end{block}
%     \begin{block}<2->{Vorteile}
%         \begin{itemize}
%             \item<2-> Geschwindigkeit
%             \item<2-> Parallelisierbar
%         \end{itemize}
%     \end{block}
% \end{frame}
