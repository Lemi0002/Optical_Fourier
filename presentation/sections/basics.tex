\section{Grundlagen}

\begin{frame}{Grundlagen - Beugung von Wellen}
    \centering
    \includegraphics[width=0.9\linewidth]{images/beugung_von_wasserwellen.png}
\end{frame}

\begin{frame}{Grundlagen - Prinzip von Huygens}
    \includegraphics[width=\linewidth]{../../SeminarHarmonischeAnalysis/buch/papers/opt/images/huygens.pdf}
\end{frame}

\begin{frame}{Grundlagen - Beugung beim JWST}
    \begin{columns}
        \begin{column}{0.5\textwidth}
            \centering
            \includegraphics[width=\textwidth, page = 1]{../../SeminarHarmonischeAnalysis/buch/papers/opt/images/jwst_sechseck.pdf}
            Beugung am Spiegel
        \end{column}
        \pause
        \begin{column}{0.5\textwidth}
            \centering
            \includegraphics[width=\textwidth, page = 2]{../../SeminarHarmonischeAnalysis/buch/papers/opt/images/jwst_sechseck.pdf}
            Beugung an den Streben
        \end{column}
    \end{columns}
\end{frame}

\begin{frame}[plain]
    \begin{tikzpicture}[overlay, remember picture]
        \node[inner sep=0pt] at (current page.center)
        {\includegraphics[height=\pdfpageheight]{../../SeminarHarmonischeAnalysis/buch/papers/opt/images/jamesWebb_publicDomain.png}};
    \end{tikzpicture}
\end{frame}

\begin{frame}{Grundlagen - Wellendarstellung}
    \begin{center}
        \includegraphics[width=0.7\linewidth]{../../SeminarHarmonischeAnalysis/buch/papers/opt/images/welle.pdf}
    \end{center}
    \begin{equation*}
        \zeta(x, t)
        =
        \zeta_0 \cdot e^{j(\omega t - \vec{k}\cdot\vec{x})}
    \end{equation*}
    \begin{equation*}
        k
        =
        \frac{\omega}{u}
        =
        \frac{2 \pi}{\lambda}
    \end{equation*}
\end{frame}

\begin{frame}{Grundlagen - Maxwell}
    \begin{columns}
        \begin{column}{0.5\textwidth}
            \begin{center}
                \includegraphics[width=\linewidth]{../../SeminarHarmonischeAnalysis/buch/papers/opt/images/maxwell.pdf}
            \end{center}
        \end{column}

        \begin{column}{0.5\textwidth}
            \pause
            \begin{block}{Block}
                \begin{align*}
                    \oint_{S=\partial V} \varepsilon\vec{E} \cdot\, d\vec{S}
                    &=
                    \int_{V}\rho\, dV
                    \\
                    \int_{0}^{a}\int_{0}^{2\pi} \varepsilon E\cdot 1 \cdot r\, d\varphi dl
                    &=
                    Q
                    \\
                    2\pi ra\varepsilon E
                    &=
                    Q
                \end{align*}
            \end{block}
            \pause
            \begin{exampleblock}{dfasdf Example block}
                \begin{equation*}
                    E(r)
                    =
                    \frac{Q}{2\pi\varepsilon a} \cdot \frac{1}{r}
                    =
                    C \cdot \frac{1}{r}
                \end{equation*}
            \end{exampleblock}
        \end{column}
    \end{columns}
\end{frame}

\begin{frame}{Grundlagen - Beugungsintegral}
    \begin{center}
        \includegraphics[width=0.7\linewidth]{../../SeminarHarmonischeAnalysis/buch/papers/opt/images/derivation.pdf}
    \end{center}
    % \begin{equation*}
    %     E(y_p, t)
    %     =
    %     C\zeta_0 \cdot \int_{-\infty}^{\infty}f(y)\cdot\frac{e^{j(\omega t - \vec{k}\cdot\vec{r})}}{r} \,dy
    % \end{equation*}
    \begin{equation}
        dE
        =
        E(r) \cdot \zeta(r, t) \cdot dy
        =
        \frac{C}{r} \cdot \zeta_0 \cdot e^{j(\omega t - \vec{k}\cdot\vec{r})} \cdot dy
    \end{equation}
\end{frame}
