\appendix

\begin{frame}{Anhang - Referenzen}
    \nocite{*} % Display all references regardless of if they were cited
    \bibliography{example.bib}
    \bibliographystyle{plain}
\end{frame}

\begin{frame}{Anhang - Intensität}
    \begin{block}{Elektrische Feldstärke ist nicht direkt sichtbar}
        \begin{align*}
            I(y_p)
            &=
            \kappa \cdot |E(y_p)|^2
            \\
            &=
            \kappa \cdot \left(\frac{C\zeta_0}{l} \cdot 1 \cdot \int_{-\infty}^{\infty}f(y)\cdot e^{-j\frac{ky_p}{l}y} \,dy\right)^2
            \\
            &=
            \kappa \cdot \frac{C^2\zeta_0^2}{l^2}\cdot \left(\int_{-\infty}^{\infty}f(y)\cdot e^{-j\frac{ky_p}{l}y} \,dy\right)^2
        \end{align*}        
    \end{block}
\end{frame}

\begin{frame}{Anhang - Wellenlängen}
    \begin{table}
        \centering % Centre the table on the slide
        \begin{tabular}{l c}
            \toprule
            Lichtfarbe & Wellenlänge in nm \\
            \toprule
            Violett    & $380 - 420$       \\
            Blau       & $420 - 490$       \\
            Grün       & $490 - 575$       \\
            Gelb       & $575 - 585$       \\
            Orange     & $585 - 650$       \\
            Rot        & $650 - 780$       \\
            \bottomrule
        \end{tabular}
        \caption{Wellenlänge von sichtbarem Licht}
    \end{table}
\end{frame}

\begin{frame}{Anhang - Bildquellen}
    \begin{block}{Beugung von Wasserwellen}
        \url{https://www.rhetos.de/html/lex/beugung_von_wasserwellen.htm}
    \end{block}

    \begin{block}{JWST}
        \url{https://webbtelescope.org/contents/media/images/2022/031/01G77PKB8NKR7S8Z6HBXMYATGJ}
    \end{block}
\end{frame}

\begin{frame}{Anhang - Präsentation}
    \begin{itemize}
        \item \LaTeX{} Präsentation mit dem Theme \emph{Focus}
        \item \url{github.com/Lemi0002/Optical_Fourier} \footnote{Commit: \StrGobbleRight{\commit}{34}}
    \end{itemize}
\end{frame}
