\section{Beugungsintegral}

\tikzstyle{my arrow} = [draw=main!80!background, very thick, single arrow, minimum height=6.5cm, shape border rotate =#1, fill=main!10!background]`

\begin{frame}{Beugungsintegral - Schlachtplan}
    \begin{columns}
    \begin{column}{0.2\textwidth}
        \centering
        \begin{tikzpicture}<1->[overlay]
            \node at (3,0) [my arrow=-180] {\rotatebox{90}{\large{Von der Beugung zu Fourier}}};
        \end{tikzpicture}
    \end{column}    

    \begin{column}{0.8\textwidth}
        
        \begin{block}<2->{Allgemeines Beugungsintegral}
            \begin{itemize}
                \item Summierung vieler Einzelwellen
            \end{itemize}
        \end{block}
        \begin{block}<3->{Fresnelapproximation}
        \begin{itemize}
            \item Für kleine Winkel
        \end{itemize}
    \end{block}
    \begin{block}<4->{Fraunhoferapproximation}
        \begin{itemize}
            \item Linearisierung in $y$
        \end{itemize}
    \end{block}
    \begin{exampleblock}<5->{Fourierintegral}
        \begin{itemize}
            \item Optische Fouriertransformation
        \end{itemize}
    \end{exampleblock}
\end{column}
\end{columns}
\end{frame}

\begin{frame}{Beugungsintegral - Allgemein}
    \begin{center}
        \includegraphics[width=0.6\linewidth]{../../SeminarHarmonischeAnalysis/buch/papers/opt/images/derivation.pdf}
    \end{center}
    \begin{block}{Geometrie}
        \begin{equation*}
            r
            =
            \sqrt{l^2 + (y_p-y)^2}
            =
            l \sqrt{1 + \frac{(y_p-y)^2}{l^2}}
        \end{equation*}
    \end{block}
\end{frame}

\begin{frame}{Beugungsintegral - Allgemein}
    \begin{columns}
        \begin{column}{0.475\textwidth}
            \centering
            \includegraphics[width=\linewidth]{../../SeminarHarmonischeAnalysis/buch/papers/opt/images/derivation.pdf}
        \end{column}

        \begin{column}<1->{0.475\textwidth}
            \begin{block}<1->{Geometrie}
                \begin{equation*}
                    r
                    =
                    \sqrt{l^2 + (y_p-y)^2}
                    =
                    l \sqrt{1 + \frac{(y_p-y)^2}{l^2}}
                \end{equation*}
            \end{block}
            \begin{block}<1->{Viele kleine Einflüsse}
                \begin{equation*}
                    dE
                    =
                    E(r) \cdot \zeta(r, t) \cdot dy
                    =
                    \frac{C}{r} \cdot \zeta_0 \cdot e^{j(\omega t - \vec{k}\cdot\vec{r})} \cdot dy
                \end{equation*}
            \end{block}
        \end{column}
    \end{columns}
    \begin{block}<2->{Integriert über die Blende}
        \begin{equation*}
            E(y_p, t)
            =
            \int_{y_b}^{y_b + b} E(r) \cdot \zeta(r, t) \cdot \,dy
            =
            \int_{y_b}^{y_b+b}C\zeta_0 \cdot \frac{e^{j(\omega t - \vec{k}\cdot\vec{r})}}{r} \,dy
            \end{equation*}
    \end{block}
\end{frame}

\begin{frame}{Beugungsintegral - Allgemein}
    \begin{columns}
        \begin{column}[t, onlytextwidth]{0.475\textwidth}
            \begin{block}<1->{\strut Integriert über die Blende (cont.)}
                \begin{equation*}
                    E(y_p, t)
                    =
                    C\zeta_0 \cdot \int_{y_b}^{y_b+b}\frac{e^{j(\omega t - \vec{k}\cdot\vec{r})}}{r} \,dy
                    \end{equation*}
            \end{block}
        \end{column}
        \begin{column}[t, onlytextwidth]{0.475\textwidth}
            \begin{block}<2->{\strut Blendenfunktion anstatt ein einzelner Spalt}
                \begin{equation*}
                    f(y)
                    \in
                    [0, 1]
                    \end{equation*}
            \end{block}
        \end{column}
    \end{columns}
    \begin{exampleblock}<3->{Allgemeines Beugungsintegral}
        \begin{equation*}
            E(y_p, t)
            =
            C\zeta_0 \cdot \int_{-\infty}^{\infty}f(y)\cdot\frac{e^{j(\omega t - \vec{k}\cdot\vec{r})}}{r} \,dy
        \end{equation*}
        \begin{itemize}
            \item<4-> \textcolor{alert}{Nicht in geschlossener Form auflösbar}
            \item<5-> Fresnel- und Fraunhofer-Approximationen
        \end{itemize}
    \end{exampleblock}
    % \strut \only<4->{\alert<4->{Nicht fundamental auflösbar} $\Longrightarrow$ Fresnel \& Fraunhofer Approximationen}  
\end{frame}

\begin{frame}{Beugungsintegral - Fresnel}
    \begin{columns}
        \begin{column}[onlytextwidth]{0.475\textwidth}
            \centering
            \includegraphics[width=\linewidth]{../../SeminarHarmonischeAnalysis/buch/papers/opt/images/derivation.pdf}
        \end{column}

        \begin{column}[onlytextwidth]<1->{0.475\textwidth}
            \begin{alertblock}<1->{Bedingung für die Fresnel-Approximation}
                \begin{equation*}
                    y, y_p
                    \ll
                    l
                \end{equation*}
            \end{alertblock}
            \begin{block}<2->{Binominalexpansion}
                \begin{equation*}
                    (1 + \varepsilon)^n
                    \approx
                    1 + n\varepsilon
                \end{equation*}
                \centering
                Gültig für $\varepsilon \ll 1$
            \end{block}
        \end{column}
    \end{columns}
    \begin{block}<3->{Abstand Quelle zu Auswertungspunkt}
        \begin{equation*}
            r
            =
            l \sqrt{1 + \frac{(y_p-y)^2}{l^2}}
            \approx
            l \left(1 - \frac{(y_p-y)^2}{2l^2}\right)
            =
            l - \frac{(y_p-y)^2}{2l}
        \end{equation*}
    \end{block}
\end{frame}

\begin{frame}{Beugungsintegral - Fresnel}
    \begin{columns}
        \begin{column}[t, onlytextwidth]<1->{0.475\textwidth}
            \begin{block}<1->{\strut Allgemeines Beugungsintegral}
                \begin{equation*}
                    E(y_p, t)
                    =
                    C\zeta_0 \cdot \int_{-\infty}^{\infty}f(y)\cdot\frac{e^{j(\omega t - \vec{k}\cdot\alert{\vec{r}})}}{\alert{r}} \,dy
                \end{equation*}
            \end{block}
        \end{column}
        \begin{column}[t, onlytextwidth]<1->{0.475\textwidth}
            \begin{block}<1->{\strut Abstand Quelle zu Auswertungszeitpunkt}
                \begin{equation*}
                    r
                    =
                    l - \frac{(y_p-y)^2}{2l}
                \end{equation*}
            \end{block}
        \end{column}
    \end{columns}

    \begin{block}<2->{Vereinfachtes Beugungsintegral}
        \begin{equation*}
            E(y_p, t)
            =
            C\zeta_0 \cdot e^{j\omega t} \cdot e^{-jkl} \cdot \int_{-\infty}^{\infty}f(y)\cdot\frac{e^{jk\frac{(y_p-y)^2}{2l}}}{l - \frac{(y_p-y)^2}{2l}} \,dy
        \end{equation*}
        \begin{itemize}
            \item<3-> Nenner immer noch zu komplex
            \item<4-> Soll zum Zeitpunkt $t=0$ berechnet werden
        \end{itemize}

    \end{block}
\end{frame}

\begin{frame}{Beugungsintegral - Fresnel}
    \begin{columns}
        \begin{column}<1->{0.475\textwidth}
            \begin{alertblock}<1->{\strut Bedingungen}
                \begin{align*}
                    (y_p - y)^2
                    &\ll
                    l \qquad\qquad k
                    =
                    \frac{2\pi}{\lambda}
                    \gg
                    1
                    \\
                    t
                    &=
                    0
                \end{align*}
            \end{alertblock}
        \end{column}
        \begin{column}<1->{0.475\textwidth}
            \begin{block}<1->{\strut Nenner vereinfachen}
                \begin{equation*}
                    r
                    =
                    l - \frac{(y_p-y)^2}{2l}
                    \approx
                    l
                \end{equation*}
            \end{block}
        \end{column}
    \end{columns}

    \begin{block}<2->{Vereinfachtes Beugungsintegral}
        \begin{equation*}
            E(y_p, t)
            =
            C\zeta_0 \cdot \alert{e^{j\omega t}} \cdot e^{-jkl} \cdot \int_{-\infty}^{\infty}f(y)\cdot\frac{e^{jk\frac{(y_p-y)^2}{2l}}}{\alert{l - \frac{(y_p-y)^2}{2l}}} \,dy
        \end{equation*}
    \end{block}
    \begin{exampleblock}<3->{Fresnel-Beugungsintegral}
        \begin{equation*}
            E(y_p)
            =
            \frac{C\zeta_0}{l} \cdot e^{-jkl} \cdot \int_{-\infty}^{\infty}f(y)\cdot e^{jk\frac{(y_p^2 - 2y_py + y^2)}{2l}} \,dy
        \end{equation*}
        \begin{itemize}
            \item<4-> Komplizierter Exponent $\Rightarrow$ Fraunhofer-Approximation
        \end{itemize}

    \end{exampleblock}
\end{frame}

\begin{frame}{Beugungsintegral - Fraunhofer}
    \begin{columns}
        \begin{column}[onlytextwidth]{0.47\textwidth}
            \centering
            \includegraphics[width=\linewidth]{../../SeminarHarmonischeAnalysis/buch/papers/opt/images/derivation.pdf}
        \end{column}
        \begin{column}[onlytextwidth]<1->{0.47\textwidth}
            \begin{alertblock}<1->{Bedingung für Fraunhofer-Approximation}
                \begin{equation*}
                    y
                    \ll
                    y_p
                    \ll
                    l
                \end{equation*}
            \end{alertblock}
            \begin{block}<2->{Linearisierung in y}
                \begin{equation*}
                    y_p^2 - 2y_py + y^2 \approx y_p^2 - 2y_py
                \end{equation*}
            \end{block}
        \end{column}
    \end{columns}
    \begin{block}<3->{Einsetzen in die Fresnel Approximation}
        \begin{align*}
            E(y_p)
            &=
            \frac{C\zeta_0}{l} \cdot e^{-jkl} \cdot \int_{-\infty}^{\infty}f(y)\cdot e^{jk\frac{(\alert{y_p^2 - 2y_py + y^2})}{2l}} \,dy
            \\
            &\approx
            \frac{C\zeta_0}{l} \cdot e^{-jkl} \cdot \int_{-\infty}^{\infty}f(y)\cdot e^{jk\frac{(y_p^2 - 2y_py)}{2l}} \,dy
        \end{align*}
    \end{block}
\end{frame}

\begin{frame}{Beugungsintegral - Fraunhofer}
    \begin{block}<1->{Einsetzen in die Fresnel Approximation (cont.)}
        \begin{align*}
            E(y_p)
            &\approx
            \frac{C\zeta_0}{l} \cdot e^{-jkl} \cdot \int_{-\infty}^{\infty}f(y)\cdot e^{jk\frac{(y_p^2 - 2y_py)}{2l}} \,dy
            \\
            &=
            \frac{C\zeta_0}{l} \cdot e^{-jkl} \cdot e^{jk\frac{y_p^2}{2l}} \cdot \int_{-\infty}^{\infty}f(y)\cdot e^{-jk\frac{y_py}{l}} \,dy
            \\
            &=
            \frac{C\zeta_0}{l} \cdot e^{-jk\left(l-\frac{y_p^2}{2l}\right)} \cdot \int_{-\infty}^{\infty}f(y)\cdot e^{-j\frac{ky_p}{l}y} \,dy
        \end{align*}
    \end{block}
    \begin{exampleblock}<2->{Fouriertransformation der Blendenfunktion}
        \begin{align*}
            E(y_p)
            &=
            C \cdot \int_{-\infty}^{\infty}f(y)\cdot e^{-jy_py} \,dy
        \end{align*}
    \end{exampleblock}
\end{frame}
